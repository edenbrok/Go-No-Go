\section{Research Approach}

\subsection{Disinformation is a post-normal problem}
%what is post-normal science?
%why is disinfo a post-normal problem?


\subsection{A Metamodel to enable theoretical exploration}

%what is theoretical exploration and how can we do it?

%what is a metamodel?
%metamodels
Three types (?). Type 2: A type model (generalization) of a token (specific instance) model:

If a model is an abstraction of reality for some given purpose, a meta-model describes the next level of abstraction: It describes the structure and behaviour of a class of models and is not case-specific (Sprinkle et al. 2010). Metamodeling is commonly used in software development to reduce the resulting software's sensitivity towards change (Atkinson \& Kühne, 2003). UML is an example of a metamodel - with its rules and language specific models can be created. \\

In simulation modeling, meta-models can be used to describe how different (sub)components of a model relate or communicate to each other (Cetinkaya 2010). 


This purpose is not so very different from that of exploratory modeling as described by Bankes (1993). He proposed exploratory modeling as an approach in cases where there is insufficient or uncertain knowledge about the system of interest. Exploratory modeling explores the consequences of changing assumptions or mechanisms through simulation experiments.

SO WHY IS IT DIFFERENT HUH?

%why use metamodel to do theoretical exposition?
%allows you to be explicit about the post-normalness without starting from absolute zero.

\subsection{Research Goal \& Questions}