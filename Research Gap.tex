\section{Research Gap}

\begin{itemize}
	\item \textbf{Dependence on scope \& contextual factors:} Psychological studies showed that individual beliefs and skills impact if people fall for disinformation, network studies show that the inclusion of multiple networks can change diffusion behaviour and even enable information cascades where they were not possible, and a simulation study found that changing the assumptions on how people share opinions determines the outcomes on a system level. Yet most literature does not (explicitly) address this dependence on context and sensitivity to scoping when it comes to disinformation. 
	\item \textbf{A variety of plausible (theoretical) explanations:} Explanations of why disinformation is so dangerous range from individual skills in media literacy, psychological biases we are all subject to, to features of the online platforms over which it is spread (i.e. bots, algorithms that encourage extremism), to broader social and economic trends such as a lack of trust in institutions and (economic) inequality. Relating to the previous point, which explanation is most appropriate is likely dependent on context. There is, however, no comprehensive work to explain what this variety of explanations means when trying to understand the problem on online disinformation.
	\item \textbf{Lack of a system-level, multidisciplinary/transdisciplinary approach:} The previous two points also strongly relate to this observation: There is very little work that tries to comprehensively bring together work on disinformation from various disciplines to understand how we may understand the problem on a system level (papers that do this are \cite{lewandowsky2017beyond} and \cite{Starbird2019}).
\end{itemize}

