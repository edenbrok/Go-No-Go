\section{Methodology}
After the establishing the boundaries and criteria of the metamodel, the research becomes iterative - that is, the experience obtained in creating a simulation model will feed into both the metamodel and subsequent simultation models, and exploratory analysis of a simulation model can inform areas of focus in the next case study.

%For each section:
%What is the goal for this section/what question does it answer?
%Define the underlying theory - why is it approriate for my goal?
%Describe the steps of the method
%Describe the tools I am going to use.
%What is the outcome?

\subsection{Defining the purpose, scope and criteria of the metamodel}
%GOAL: See subsect heading
%UNDERLYING idea: There is no "true" theory of disinformation, so I am creating an artefact as a tool for sensemaking & theoretical exploration. So we want it do be able to do certain things and have certain properties. This is typical of a design approach
%TOOLS: Ontologies? Peer groups?

\begin{itemize}
	\item Design science approach
	\item Formulation of requirements, scope and goals
	\item Possibly: participatory approach in order to define the scope of the model \footnote{Inspired by the peer communities from PNS, the fact that we as well-educated, well-represented citizens probably cannot fully understand why people buy into (political) disinfo, and the fact that according to that one paper I fall into the conspiracy-theory believing group.}. 
\end{itemize}

\subsection{Finding the elements of the metamodel through case-study based simulation modelling}
%GOAL: To further develop/instantiate the metamodel/theory by case study
%UNDERLYING: Generative modelling: If you didn't grow it, you didn't explain it
%UNDERLYING: Simulation modelling: Can meaningfully use them to study certain effects. Wrong, but helpful.
%Three case studies with different focus/level of detail
%BLM/anti-BLM protest FB events - causality of polarization/disinfo and the role of digital infrastructure (spread?)
%MH17 - Goals & societal effects - is the purpose of disinfo for us to believe it? What societal effects do you want to create (content?)
%Greenwashing - Propaganda theory of media (Herman Chomsky) & flak - how do long-term campaigns work & how do they use the internet? (effect?)
%TOOLS: Case study, sim modelling
	
\begin{itemize}
	\item Three case studies across different contexts, i.e. Russian disinformation campaign around MH-17, the 5G/Corona conspiracy, and long-term greenwashing by fossil fuel companies
	\item Data collection through mixed methods, i.e. interviews/focus groups (or a corona-proof alternative), document analysis, network analysis, data from crawling social media posts.
	\item Simulation model: Likely primarily ABM (i.e. NetLogo), but other modeling paradigms possible as the metamodel or the case demands.
\end{itemize}
%simulation modelling using ABM

\subsection{Analysing the importance and sensitivity of metamodel elements, policy analysis}
\begin{itemize}
	\item Sensitivity analysis and assumption testing (exploratory modeling), using the ema-workbench
	\item Policy analysis (either exisiting suggestions to combat disinformation or based on previous analysis)
	\item Inter-model analysis: Once I have multiple simulation models of different contexts, also compare differences in element importance/dominance and effect of assumptions across models.
\end{itemize}

\subsection{Evaluation of the metamodel}
%GOAL: Evaluation and connection to reality? 
\begin{itemize}
	\item Evaluate the metamodel on the basis of the criterea established previously
	\item Analysis of the results obtained by the cases - what can they tell us about online disinformation on a higher level?
	\item Answering the ``Now what?'' question
\end{itemize}